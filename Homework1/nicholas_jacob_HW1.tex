\documentclass[11pt]{article}
\usepackage{hyperref}
\usepackage{amsthm}
\usepackage{amsmath}
\usepackage{amsfonts}
\usepackage{tikz}
\usepackage{ wasysym }

\newtheorem{example}{Example}


\author{}
\title{}

\begin{document}
%\maketitle
{\Large
%Change Document name to: Graded Homework 1\_Jacob\_Nicholas
\noindent NAME:  Nicholas Jacob\\ 
STUDENT ID: \# 113578513\\
HOMEWORK NUMBER: 1\\
COURSE: DSA 5113 Advanced Analytics and Metaheuristics\\ 
SECTION: ONLINE\\SEMESTER: Spring 2024\\
INSTRUCTOR:  DR. Charles Nicholson\\
 SCORE:}

\newpage
\begin{enumerate}
\item To examine this story, I will create a table.  I will assume that a sinister lying creature always lies (even though we know this too not be true eg. politicians).  So really this lays out two circumstances with two options in each.  

\begin{tabular}{l|l|l}
&gurump&pvlork\\ \hline
Jedi&No& Yes\\
Jedi& Yes&No\\
Sith &No & Yes\\
Sith & Yes &No

\end{tabular}

Let's examine each with our information.  For the first we see that since Jedi would tell the truth, it cannot be.  You would not answer yes to a wrong question.  Similarly you cannot have the second either as you will tell the truth.  We move on to sith liars.  If gurump means No and asked if it means yes, a liar would reply in the affirmative.  This means the logic on the third one is correct.  We see the fourth is also possible.  Since gurump is yes and the creature lies, we would answer no or pvlork.  Thus we see that Baby Yoda is a liar.  We cannot however determine the meaning of gurump and pvlork.

\begin{tabular}{l|l|l|l}
&gurump&pvlork&Possible\\ \hline
Jedi&No& Yes& No\\
Jedi& Yes&No& No\\
Sith &No & Yes& Yes\\
Sith & Yes &No&Yes

\end{tabular}

\end{enumerate}



\end{document}