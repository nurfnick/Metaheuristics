\documentclass[11pt]{article}
\usepackage{hyperref}
\usepackage{amsthm}
\usepackage{amsmath}
\usepackage{amsfonts}
\usepackage{tikz}
\usepackage{ wasysym }

\newtheorem{example}{Example}


\author{}
\title{}

\begin{document}
%\maketitle
{\Large
%Change Document name to: Graded Homework 1\_Jacob\_Nicholas
\noindent NAME:  Nicholas Jacob\\ 
STUDENT ID: \# 113578513\\
HOMEWORK NUMBER: 1\\
COURSE: DSA 5113 Advanced Analytics and Metaheuristics\\ 
SECTION: ONLINE\\SEMESTER: Spring 2024\\
INSTRUCTOR:  DR. Charles Nicholson\\
 SCORE:}

\newpage
\begin{enumerate}
\item To examine this story, I will create a table.  I will assume that a sinister lying creature always lies (even though we know this too not be true eg. politicians).  So really this lays out two circumstances with two options in each.  

\begin{tabular}{l|l|l}
&gurump&pvlork\\ \hline
Jedi&No& Yes\\
Jedi& Yes&No\\
Sith &No & Yes\\
Sith & Yes &No

\end{tabular}

Let's examine each with our information.  For the first we see that since Jedi would tell the truth, it cannot be.  You would not answer yes to a wrong question.  Similarly you cannot have the second either as you will tell the truth.  We move on to sith liars.  If gurump means No and asked if it means yes, a liar would reply in the affirmative.  This means the logic on the third one is correct.  We see the fourth is also possible.  Since gurump is yes and the creature lies, we would answer no or pvlork.  Thus we see that Baby Yoda is a liar.  We cannot however determine the meaning of gurump and pvlork.

\begin{tabular}{l|l|l|l}
&gurump&pvlork&Possible\\ \hline
Jedi&No& Yes& No\\
Jedi& Yes&No& No\\
Sith &No & Yes& Yes\\
Sith & Yes &No&Yes

\end{tabular}

\item I am going to state the problem here
\begin{quote}
A portfolio manager in charge of a bank portfolio has \$10 million to invest. The
securities available for purchase, as well as their respective quality ratings, maturities, and yields, are shown
in Table

\begin{tabular}{c|c|c|c|c|c|c}
Name & Type & QS Moody's & QS Banks& Years to M & Yield to m& After-tax yield\\\hline
A& Municipal& Aa &2& 9& 4.3\% &4.3\%\\
B& Agency& Aa& 2& 15& 5.4& 2.7\\
C& Government& Aaa &1 &4 &5.0 &2.5\\
D &Government &Aaa& 1& 3& 4.4& 2.2\\
E& Municipal &Ba &5 &2 &4.5 &4.5
\end{tabular}

The bank places the following policy limitations on the portfolio manager’s actions:
\begin{enumerate}
\item Government and agency bonds must total at least \$4 million.
\item  The average quality of the portfolio cannot exceed 1.4 on the bank’s quality scale. (Note that a low
number on this scale means a high-quality bond.)
\item  The average years to maturity of the portfolio must not exceed 5 years.
\end{enumerate}
Assuming that the objective of the portfolio manager is to maximize after-tax earnings and that the tax rate is 50 percent, what bonds should he purchase? If it became possible to borrow up to \$1 million at 5.5 percent before taxes, how should his selection be changed?
\end{quote}

\end{enumerate}



\end{document}